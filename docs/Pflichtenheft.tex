\documentclass[12pt,titlepage,german,a4]{article}

\begin{document}
    \title{\bf Miniprojekt: Minimax-Maschine \\ Pflichtenheft}
    \date{Hardware Projekt 2017 \\ 03. Januar 2017}
    \author{Maximilian Lumpe, Niklas Blume, Jan Feuchter, Phu Bac Duong}
    \maketitle

    \tableofcontents

    \newpage

    \section{Einleitung}
    In diesem Miniprojekt im Rahmen des Hardware-Praktikums besch{\"a}ftigen wir uns mit der Minimax-Maschine, welche uns grundlegend aus der Vorlesung "Grundlagen der Rechnerarchitektur" bekannt ist. Zur L{\"o}sung der Aufgaben ist es hierbei notwendig, die vorgegebene Grundstruktur der Maschine geeignet zu erweitern, um die Algorithmen zu realisieren.\\Die Vorbereitung auf unser Projekt wird dokumentiert und strukturiert durch das von uns erstellte Pflichtenheft. Das Pflichtenheft wird nur unsere Vorbereitung beinhalten. Die Ergebnisse werden in einer weiteren Dokumentation enthalten sein.

    \section{Aufgabe: Paketanalyse}
    Nach unserem Verst{\"a}ndniss ist das Ziel der Aufgabenstellung das Implementieren des Algorithmus "Paketanalyse" auf der Minimax-Maschine. Dieser Algorithmus wertet die L{\"a}nge des Nutzdatenteils der Datenpakete aus dem Speicher der Maschine aus.\\Jedes Paket besteht aus einem Header mit 80 Bits, gefolgt von dem Datenteil mit variabler L{\"a}nge. Ein Paket beginnt mit dem festgelegten Bitmuster 1110. Der Header enth{\"a}lt eine 2 Bytes lange Kanalnummer, die bei der Bitstelle 32. beginnt. Zu einem Kanal geh{\"o}ren mehrere Datenpakete mit einer eindeutigen Kanalnummer. Die Anzahl der Bits, die in den Speicher geladen werden, wird als bekannt vorausgesetzt und wird in ein entsprechendes Register vorgeladen.\\Nun soll der "Paketanalyse"-Algorithmus eine Tabelle, die Kanalnummern und zugeh{\"o}rige Datenl{\"a}ngen (in Bits) enth{\"a}lt, anlegen. Haben mehrere Paket dieselbe Kanalnummer, so werden die L{\"a}ngen des Nutzdatenteils zusammengef{\"u}gt. Die Tabelle soll ab einer beliebigen Speicheradresse au{\ss}erhalb des Paketfeldbereichs im Hauptspeicher der Maschine abgelegt werden.\\Diese Aufgabenstellung soll mit dem gegebenen Minimax-Simulator simuliert und getestet werden. Die Maschine kann durch vorgegebene Bauteile erweitert werden, was sich jedoch auf die Bewertung auswirkt. Der Algorithmus wird in Form der Steuertabelle implementiert und soll au{\ss}erdem als Flussdiagramm abgegeben werden.

    \newpage

    \section{Ist-Analyse der Basismaschine}
    Die Minimax-Maschine ist ein minimales Rechensystem welches im Wesentlichen aus einigen Registern (Basis ACCU, PC, IR, MDR, MAR) und einer arithmetisch-logischen Einheit (ALU), welche den Datenpfad bilden, und einem Hauptspeicher (HS) besteht. Jeder Register hat als Eingang mindesten die ALU und zus{\"a}tzlich einen Eingang, der den Schreibzugriff regelt. Der Befehlsablauf des Systems wird {\"u}ber ein Mikroprogramm festgelegt.\\
    F{\"u}r das Projekt kann die Architektur der Minimax-Maschine um zuss{\"a}tzliche Register bzw. Sign Extension Units erweitert werden. \\
    \begin{itemize}
        \item Die Basismaschine:
        \begin{enumerate}
            \item ACCU: Abk{\"u}rzung f{\"u}r “Accumulator“ ein Zwischenspeicher, um mit MDR Operationen durchf{\"u}hren zu k{\"o}nnen.
            \item PC: Abk{\"u}rzung f{\"u}r “program counter“, enth{\"a}lt den Programmz{\"a}hler, welcher den n{\"a}chsten Befehl beeinflusst.
            \item IR: Abk{\"u}rzung f{\"u}r “instruction register“, enth{\"a}lt Opcode(8 Bit) und Adressteil(24 bit).
            \item MDR: Abk{\"u}rzung f{\"u}r “memory data register“, enth{\"a}lt je nach Einstellung des Multiplexers textttMDR.Sel, verschiede Daten entweder aus Hauptspecher oder aus der ALU.
            \item MAR: Abk{\"u}rzung f{\"u}r “memory adress register“, enth{\"a}lt die Speicheradresse an der ausdem Hauptspeicher Daten geladen oder geschrieben werde sollen.
            \item HS: Abk{\"u}rzung f{\"u}r “Hauptspeicher“, er wird mit 24-Bit durch MAR adressiert und gibt eine 32-Bit Zahl zur{\"u}ck. Auf die selbe Art funktioniert schreiben einer 32-Bit Zahl nat{\"u}rlich auch.
        \end{enumerate}
        \item ALU-Operationen der Basismaschine:
        \begin{enumerate}
            \item ADD: Addiert ALU-Eingang A und ALU-Eingang B
            \item SUB.B: Subtrahiert ALU-Eingang A von ALU-Eingang B
            \item TRANS.A: Schaltet den ALU-Eingang A durch
            \item TRANS.B: Schaltet den ALU-Eingang B durch
        \end{enumerate}
    \end{itemize}
    Die m{\"o}glichen Operationen auf die in der ALU implementieren Operationen sind beschr{\"a}nkt ( Basis :  ADD, SUB, TRANS.A, TRANS.B ) . Die ALU kann aber mit zus{\"a}tzliche Operationen , wie z.B dem DIV-Befehl oder dem AND-Befehl , erg{\"a}nzt werden .
    Um eine Operation auszuf{\"u}hren m{\"u}ssen {\"u}ber die Multiplexer ALUSel.A und AluSel.B zwei
    Operanden ausgew{\"a}hlt werden und der ALU muss {\"u}ber die ALU Ctrl-Leitung der Code
    f{\"u}r die Operation {\"u}bergeben werden. An den Multiplexern liegen sowohl Konstanten als
    auch die Register an, welche zur ALU durchgeschaltet werden k{\"o}nnen. Das Ergebnis der
    Operation kann entweder in einem Register ({\"u}ber MDR) oder im HS (Adresse im Register
    MAR) gespeichert werden. Zus{\"a}tzlich k{\"o}nnen Flags (Basis nur ein Flag ALU RESULT ==
    0) gesetzt werden, welche zur{\"u}ck zur Control Unit (CU) geleitet werden um z. B. bedingte
    Spr{\"u}nge auszuf{\"u}hren. \\
    Die uns vorliegende Minimax-Maschine arbeitet mit 32-Bit und speichert Werte mit
    32-Bit in den Registern und im HS. Alle ALU-Operationen werden folglich mit 32-Bit ausgef{\"u}hrt. Dies stellt sich jedoch f{\"u}r unser Aufgabe als Hindernis, da wir die Daten
    bitweise untersuchen m{\"u}ssen, Daten aus dem HS und den Registern jedoch nur als 32-Bit
    Zahlen auslesen k{\"o}nnen und nicht als einzelne Bits.\\
    Folglich wird die Basismaschine um einige Konstanten, Operationen und Register erweitert
    werden m{\"u}ssen, welche im “Implementierungskonzept“ n{\"a}her aufgef{\"u}hrt sind.

    \section{Beschreibung des Implementierungskonzeptes}

    \section{Angestrebte Projektergebnisse}
        \begin{enumerate}
            \item erster Stichpunkt
            \item zweiter Stichpunkt
            \item dritter Stichpunkt
        \end{enumerate}
    \section{Arbeitsplanung und -verteilung}

    \section{Anhang: Flussdiagramme geplanter Mikroprogramme}

    \section{Anhang: Verwendete Hilfsmittel}


\end{document}
